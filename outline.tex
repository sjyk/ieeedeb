%% LyX 2.0.6 created this file.  For more info, see http://www.lyx.org/.
%% Do not edit unless you really know what you are doing.
\documentclass[english]{article}
\usepackage[T1]{fontenc}
\usepackage[latin9]{inputenc}
\usepackage{enumitem}
\usepackage{sidecap}
\usepackage{framed}
\usepackage{cprotect}
\usepackage{enumitem}
\usepackage{listings}
\usepackage{amstext}
\usepackage{amstext}
\usepackage{pdfpages}
\usepackage{alltt}
\usepackage{epstopdf}
\usepackage{xspace,colortbl}
\usepackage[USenglish]{babel}
\usepackage{multirow}
\usepackage{url}
\usepackage{subfigure}
\usepackage{graphicx}%%
\usepackage{amssymb}
\usepackage{fmtcount}
\usepackage{amsfonts}
\usepackage{xspace}
\usepackage{amsmath}
\usepackage{multirow}
\usepackage[mathscr]{eucal}
%\usepackage{psfrag}
\usepackage{colortbl}

\usepackage{bm}
\usepackage{times}
\usepackage[nospace]{cite}
\usepackage{csquotes}
\usepackage{enumitem}
\usepackage{geometry}
\geometry{verbose,tmargin=2.54cm,bmargin=2.54cm,lmargin=2.54cm,rmargin=2.54cm}
\usepackage{babel}
\begin{document}

\newtheorem{theorem}{Theorem}
\newtheorem{example}{Example}
\newtheorem{definition}{Definition}
\newtheorem{problem}{Problem}
\newtheorem{property}{Property}
\newtheorem{proposition}{Proposition}
\newtheorem{lemma}{Lemma}
\newtheorem{corollary}{Corollary}

\newcommand{\cond}{\textrm{pred}\xspace}
\newcommand{\dataset}{data set\xspace}
\newcommand{\datasets}{data sets\xspace}
\newcommand{\spview}{\textsf{SPView}\xspace}
\newcommand{\fjview}{\textsf{FJView}\xspace}
\newcommand{\aggview}{\textsf{AggView}\xspace}
\newcommand{\hashfunc}[1]{\textsf{hash}(#1)\xspace}
\newcommand{\hashop}{\textsf{hash}\xspace}
\newcommand{\nsc}{\textsf{NormalizedSC}\xspace}
\newcommand{\rsc}{\textsf{RawSC}\xspace}

\newcommand{\avgfunc}{\ensuremath{\texttt{avg} }\xspace}
\newcommand{\maxfunc}{\ensuremath{\texttt{max} }\xspace}
\newcommand{\minfunc}{\ensuremath{\texttt{min} }\xspace}
\newcommand{\histfunc}{\ensuremath{\texttt{histogram\_numeric} }\xspace}
\newcommand{\countfunc}{\ensuremath{\texttt{count}}\xspace}
\newcommand{\sumfunc}{\ensuremath{\texttt{sum} }\xspace}
\newcommand{\varfunc}{\ensuremath{\texttt{var} }\xspace}
\newcommand{\stdfunc}{\ensuremath{\texttt{std} }\xspace}
\newcommand{\covfunc}{\ensuremath{\texttt{cov} }\xspace}
\newcommand{\corrfunc}{\ensuremath{\texttt{corr} }\xspace}
\newcommand{\medfunc}{\ensuremath{\texttt{median} }\xspace}
\newcommand{\percfunc}{\ensuremath{\texttt{percentile} }\xspace}
\newcommand{\havingfunc}{\ensuremath{\texttt{HAVING} }\xspace}
\newcommand{\selectfunc}{\ensuremath{\texttt{select} }\xspace}
\newcommand{\ratio}{\ensuremath{\rho }\xspace}


\newcommand{\insertion}{\ensuremath{\texttt{INSERT} }\xspace}
\newcommand{\update}{\ensuremath{\texttt{UPDATE} }\xspace}
\newcommand{\delete}{\ensuremath{\texttt{DELETE} }\xspace}

\author{ Sanjay Krishnan, Jiannan Wang, Ken Goldberg, Michael J. Franklin \\
AMPLab, UC Berkeley\\
\{jnwang, sanjaykrishnan, franklin, goldberg\}@berkeley.edu
%\email{milo@cs.tau.ac.il,~~~~tim\_kraska@brown.edu}
}

\title{SampleClean: Fast and Reliable Analytics on Dirty Data}
\maketitle
\begin{abstract}
A perennial challenge in data analytics is presence of dirty data
in the form of missing, duplicate, incorrect or inconsistent values.
Data analysts report that data cleaning remains one of the most time
consuming steps in the analysis process, and data cleaning can require
a significant amount of developer effort in writing software or rules
to fix the corruption. Since \emph{a priori}, the magnitude of data
error is unknown, any amount of error is potentially biasing, and
this presents a dichotomy between facing the cost of data cleaning
or coping with consequences of unknown inaccuracy. In this article,
we present a line of work called SampleClean, which allows the analyst
to clean a sample of data and estimate the results (with provable
guarantees) of some types of analytics--building on results in Approximate
Query Process (AQP). We describe three projects: SampleClean, Stale
View Cleaning, and ActiveClean. SampleClean returns bounded estimates
of SUM, COUNT, and AVG queries using a sample of cleaned data. Stale
View Cleaning models staleness in Materialized Views as a type of
data error and applies an extension of SampleClean for scalable incremental
maintenance. ActiveClean extends SampleClean to a class of analytics
that we call convex data analysis (subsuming common aggregate queries
and including Machine Learning). Finally, we describe a number of
open problems and future directions for the project.
\end{abstract}

\section{Introduction}
Data are susceptible to various forms of corruption such as missing,
incorrect, or inconsistent representations \cite{Gartner}.
Real-world data is commonly integrated from multiple sources, and the integration process may lead to a variety of data errors~\cite{DBLP:journals/pvldb/DongS13}. 
Data analysts report that data cleaning remains one of the most time
consuming steps in the analysis process \cite{nytimes}.
Identifying and fixing data error often requires manually inspecting data, which can quickly become costly and time-consuming. 
While crowdsourcing is an increasingly viable option for some types of errors ~\cite{DBLP:conf/sigmod/JefferyFH08,DBLP:journals/pvldb/FanLMTY10,DBLP:journals/pvldb/YakoutENOI11, gokhale2014corleone, park2014crowdfill, sampleclean,chu2015katara}, it comes at the significant cost of additional latency and the overhead of managing human workers. 

Ignoring the effects of dirty data is potentially dangerous.
Since \emph{a priori}, the magnitude of data error is unknown, any amount of error can be biasing, and this presents a dichotomy between facing the cost of data cleaning
or coping with consequences of unknown inaccuracy.
In this article, we describe a middle ground that we call SampleClean \cite{wang1999sample}; where an analyst can clean a sample of data, and use this sample to improve an erroneous query result.
The intruiging part of SampleClean is that while the query result is approximate, this approximation error is boundable--unlike the unknown data error.
The tightness of this bound is parametrized by a flexible cleaning cost.

The case for SampleClean is analagous to the case for Approximate Query Processing \cite{DBLP:conf/icde/OlkenR92, olken1993random, garofalakis2001approximate, AgarwalMPMMS13} (AQP).
For decision problems, exploratory analysis problems, and visualization, it often suffices to return an approximate query result bounded in confidence intervals; thus saving significant processing costs.
In many common aggregates, the samples-to-accuracy trade off tends to vary as $O(\frac{1}{\sqrt{n}})$, and therefore every additional $\epsilon$ factor of accuracy costs quadratically more. 
In applications where approximation can be tolerated, sampling avoids the expensive ``last mile" of processing and timely answers facilitate improved user experiences and faster analysis.

In traditional AQP, approximation necessarily sacrifices accuracy for reduced latency. 
However, the goal of SampleClean differs from AQP, as SampleClean tradesoff data transformation cost for gradual improvements in query accuracy.
While SampleClean introduces approximation error, the data cleaning mitigates errors in query results.
There is a break-even point where a sufficient amount of data is cleaned to facilitate an accurate approximation of queries on the cleaned data, and in this sense, sampling actually improves the accuracy of the query result.

SampleClean \cite{wang1999sample} and all of its extensions \cite{krishnan2015svc}, work in the \emph{budgeted data cleaning} setting. 
An analyst is allowed to apply an expensive data transformation $C(\cdot)$ to only $k\ll N$ rows in a relation.
One solution could be to draw $k$ records uniformly at random and apply data cleaning, e.g., a direct extension of AQP.
However, data cleaning presents a number of methodological problems that make this hard.
First, $C(\cdot)$ may change the sampling statistics, for example, duplicated records are more likely to be sampled.
Next, query processing on partially clean data, i.e., a mix of dirty and clean data, can lead to unreliable results due to the well known Simpsons Paradox.
Finally, high-dimensional analytics such as Machine Learning may be very sensitive to sample size, perhaps even more so than the dirty data, and techniques for avoiding sample size dependence are required.

One the key insights is that there are two constrasting query processing techniques to address every budgeted data cleaning problem.
A direct extension of AQP can estimate the true query result based on the cleaned sample (possibly with some reweighting to account for changes in sampling statistics). 
A sample of cleaned data can also be used to correct the error in a query result over the dirty data.
There is an interesting theoretical tradeoff between these approaches, where the first approach is \emph{robust} as its accuracy is independent of the magnitude of data error, and the second approach is \emph{sample-efficient} as its accuracy is less dependent on sample size.

SampleClean established the link between AQP and expensive data transformation operations such as cleaning. 
This idea opened up a number of new research opporunties, and we applied the same principles to other domains such as Materialized View Maintenance and Machine Learning.
In this article, we highlight four projects based on the SampleClean idea:

\vspace{0.5em}
\noindent \textbf{\sampleclean \cite{wang1999sample}: } \sampleclean describes query processing approaches for estimating aggregate queries using samples of clean data. \sampleclean reweights the data to compensate for changes in sampling statistics such that the estimates are unbiased and bounded in confidence intervals.

\vspace{0.5em}
\noindent \textbf{View Cleaning \cite{krishnan2015svc}: } View Cleaning generalizes the notion of data cleaning to incluce processing out-of-date data. Staleness in materialized views (MVs) manifests itself as data error, i.e., a stale view has missing, superfluous, and incorrect rows.
Like data cleaning, eager MV maintenance is expensive, and View Cleaning models querying an MV as a budgeted data cleaning problem.
Aggregate queries are approximated from a stale MV using a small sample of up-to-date data, resulting in bounded estimates.

\vspace{0.5em}
\noindent \textbf{ActiveClean: } ActiveClean extends \sampleclean to a class of analytics problems called Convex Data Analytics (subsuming the aggregates studied in \sampleclean and including Machine Learning such as Support Vector Machines and Linear Regression). ActiveClean exploits the convex structure of the problem (i.e. gradients and feature space) to prioritize cleaning data that is likely to affect the model. ActiveClean directly integrates cleaning into the model training loop and as a result gives a bounded approximation for any cleaning budget.

\vspace{0.5em}
\noindent \textbf{Wisteria \cite{haas2015wisteria}: } In Wisteria, we implemented the ideas described in the SampleClean work in a real, distributed data cleaning system. This involved designing an algebra for data cleaning transformations and optimizations for their execution.

This article is organized as follows. Section 2 describes introduces the project and its main ideas. Section 3 overviews the basic architecture of each one of the projects. Section 4/5/6 describes \sampleclean, View Cleaning, and ActiveClean respectively. Section 7 reviews the related work in this field. In Section 8, we highlight some of the open problems and future directions of the SampleClean project. Finally, we conclude in Section 9.
\section{Background and Main Ideas}

\subsection{Data Cleaning is Often Expensive}
A number of surveys of analysts report that data cleaning is one of the most time consuming steps \cite{kandel2012enterprise, nytimes}.
A number of data cleaning frameworks have been recently proposed to address the problem of corrupted data at scale\cite{khayyat2015bigdansing, chu2015katara, sampleclean}.
As errors can be domain- or dataset-specific, data cleaning is an inherently human-driven process and can require a significant amount of developer effort in writing software or rules to fix the corruption.
Automated fixes may not be reliable and can require human confirmation \cite{DBLP:journals/pvldb/YakoutENOI11}.
One way to scale up human computation is crowdsourcing which has shown recent success in entity resolution and value filling \cite{gokhale2014corleone, park2014crowdfill, sampleclean,chu2015katara}.
However, crowdsourcing comes with the costs of significant additional latency (orders of magnitude slower than data processing) and the overhead of managing human workers.

Even for more sophisticated analytics, such as machine learning model training, cleaning is still orders of magnitude slower.
We can compare recent results in data cleaning to a model training framework like CoCoA implemented on Spark \cite{jaggi2014communication}.
Per record, BigDansing, a highly optimized automated Spark-based data cleaning system is 15.5x slower than CoCoA\footnote{For CoCoA to reach a precision of 1e-3}.
Crowd based techniques like CrowdFill \cite{park2014crowdfill} and CrowdER \cite{wang2012crowder} are over 100,000x slower per record. 

\subsection{Exploiting Application Structure}
SampleClean applies sample to clean $k\ll N$ rows in a database to address the time-scale mismatch between the analytics application (e.g., SQL query, Machine Learning, Materialized View) and data cleaning.
An important aspect of this project is how the structure and semantics of that application can be used to prioritize and budget data cleaning.
In other words, a database only needs to be sufficiently clean for the requirements of the subsequent analytics.
One such application is aggregate analytics as the endpoint of many big data analytics pipelines is an aggregation.
Even the increasingly common statistical models like Support Vector Machines and Regressions can be thought of as high-dimensional aggregates.
As widely established in AQP, aggregate analytics are widely applied in domains that are tolerant of approximation e.g., visualization or hypothesis testing.

In the initial SampleClean work, we restricted the allowed aggregate queries to \sumfunc, \countfunc, and \avgfunc with predicates and group by clauses.
In the two subsequent projects, View Cleaning and ActiveClean, we expanded the scope and the semantics of the application. 
The View Cleaning problem explores data cleaning and general aggregates on derived relations with known view definitions.
We can exploit view definition to query just as much of the base data as needed to accurately answer the aggregate query for a fixed budget.
In fact, we showed that any aggregate (beyond \sumfunc, \countfunc, and \avgfunc) that could be estimated with SAQP\cite{agarwalknowing}, could be answered estimated with the View Cleaning framework.
ActiveClean generalizes the initial work on \sumfunc, \countfunc, and \avgfunc to higher-dimensional aggregates.
We defined a class of analytics called Convex Data Analytics, and show how the convex structure of the analytics can be used to guide and prioritize data cleaning.

\subsection{Approximate Query Processing on Dirty Data}
To understand how we can integrate data cleaning and sampling, let us first understand how traditional AQP is affected by dirty data.
Sampling-based approximate query processing (SAQP) is a powerful technique that allows for fast approximate results on large datasets. 
It has been well studied in the database community since the 1990s~\cite{DBLP:conf/sigmod/HellersteinHW97,DBLP:conf/sigmod/AcharyaGPR99}, and methods such as BlinkDB~\cite{DBLP:conf/eurosys/AgarwalMPMMS13} have drawn renewed attention in recent big data research. 
An important aspect of SAQP is confidence intervals, as many types of aggregates can be bounded with techniques such as concentration inequalities (e.g., Hoeffding bounds), large-deviation inequalities (e.g., Central Limit Theorem), or empirically (e.g., bootstrap).
However, these bounds assume that the only source of error is uncertainty introduced by sampling, however, the data itself may contain errors which could also affect query results. 

Suppose, there is a relation $R$ and a uniform sample $S$.
SAQP applies a query $q$ to $S$ (possibly with some scaling $c$) to return an estimate:
\[
q(R) \approx e = c \cdot q(S)
\]
If $R$ is dirty, then there is a true relation $R_{clean}$.
\[
q(R_{clean}) \ne q(R) \approx e = c \cdot q(S)
\]
The error in $e$ has two components error due to sampling $\epsilon_s$ and error due to the difference with the cleaned relation $\epsilon_c = q(R_{clean}) - q(R)$:
\[
\mid q(R_{clean}) - e \mid \le \epsilon_s + \epsilon_c
\]

While they are both forms of query result error $\epsilon_s$ and $\epsilon_c$ are very different quantities.
$\epsilon_s$ is a random variable due to the sampling, and different samples would result in different realizations of $\epsilon_s$.
As a random variable introduced by sampling, $\epsilon_s$ can be bounded by a variety of techniques as a function of the sample size.
On the other hand, $\epsilon_c$ is deterministic, and by definition is a unknown quantity until all the data is cleaned.
Thus, the bounds returned by a typical AQP framework on dirty data would neglect $\epsilon_c$.

It is possible that $R_{clean} \ne R$ but $\epsilon_c=0$.
Consider a \sumfunc query on the relation $R(a)$, where $a$ is a numerical attribute.
If half of the rows in $R$ is corrupted with $+1$ and the other half are corrupted with $-1$, then $q(R_{clean}) = q(R)$.
In the data cleaning setting, we are interested in systematic errors where $\epsilon_c > 0$ \cite{taylor1982introduction}. 
In other words, the corruption that is correlated with the data, e.g., where every record is corrupted with a $+1$.

\subsection{Direct Estimate vs. Correction}
The key quanitity of interest in this work is $\epsilon_c$, and essentially, to be able to bound
a query result on dirty data we need to either ensure $\epsilon_c$ is 0 or bound $\epsilon_c$.


\subsection{Sampling to Improve Accuracy}
\section{System Architecture and Goals}
The subsequent sections overview the SampleClean projects, and this section describes some of the general principles, challenges, and design goals. 

\subsection{Architecture}
Figure \ref{fig:arch} describes the basic architecture of SampleClean. 

\vspace{0.5em}
\noindent\textbf{Initialization: } The system is initialized with a dirty relation $R$, a user specified data cleaning technique $C(\cdot)$, a budget $k$, and some analytic query to run $q$. SampleClean provides a framework to execute the cleaner no more than $k$ times and estimate the result of $q$.  

\vspace{0.5em}
\noindent\textbf{Analytics: } In the basic SampleClean setting, the supported queries are aggregates of the form:
\begin{alltt}
q = SELECT \textsf{f}(attrs) FROM R WHERE predicate GROUP BY attrs
\end{alltt}
In SampleClean, we consider \avgfunc, \sumfunc, \countfunc, and \varfunc.
In View Cleaning, we extend this set of queries to any queries that can be estimated with a statistical bootstrap \cite{agarwalknowing}.
Finally, in ActiveClean, we extend this work to consider Machine Learning models as aggregate queries.

\vspace{0.5em}
\noindent\textbf{Sampler: } The first component of SampleClean is the sampler. The sampler selects a sample of $k$ rows from the dirty relation $R$. For each sampled row $i$, the sampler maintains a sampling probability $p_i$. The sampler is one of the primary points for optimization in this project. In SampleClean, we used a uniform sample of $k$ rows from $R$.
In View Cleaning, we select a uniform sample of $k$ rows from a view $R$, where there is some known view definition.
And in ActiveClean, we use the analytic query $q$ to select a non-uniform sample of rows from $R$.

\vspace{0.5em}
\noindent\textbf{Cleaner: }
Given the sample of dirty rows $S_{dirty}$,  the cleaner applies the user-specified data cleaning $C(\cdot)$. There are two data cleaning models for $C(\cdot)$: \emph{row-by-row} and \emph{set-of-rows}.
Formally, a row-by-row cleaner is applied to every row in $S_dirty$ and possibly changes the sampling probability as well (e.g., deduplication):
\[
(S_{clean},p_i') = \{C(r) : \forall (r,p_i) \in (S_{dirty},p_i)\}
\]
This model captures common blocking-matching entity resolution and deduplication, extraction, filtering, and value filling.

The row-by-row cleaning model is a formalization of the costs of data cleaning where each row has the same cost to clean and this cost does not change throughout the entire cleaning session.
There are, however, some cases when cleaning the first row of a certain type of corruption is expensive but all subsequent row are cheaper.
Consider a spell checker that allows the user to fix all similarly misspelled words. 

To address this problem, we also study the set-of-rows cleaning model.
In the set-of-rows model, the cleaning function $C(\cdot)$ is not restricted to updating only the rows in the sample.
$C(\cdot)$ takes the entire dirty sample as an argument (that is the cleaning is a function of the sample), the dirty data, and updates the entire dirty data:
\[
(S_{clean}, R'_{dirty}) = C(S_{dirty},R_{dirty})
\]
we require that for every record $s \in S_{dirty}$, that record is completely cleaned after applying $C(\cdot)$, giving us $S_{clean}$.

\vspace{0.5em}
\noindent\textbf{Estimator: } The goal of the estimator is to estimate the value of $q(R_{clean})$ from the cleaned sample $S_{clean}$, the dirty sample $S_{dirty}$, and the sampling probabilities $p_i'$. The estimator possibly has to modify $q$ to account for scaling differences when running the query on the sample.

\begin{SCfigure}
\includegraphics[width=.5\columnwidth]{figs/arch.png}
\caption{Given a user-specified data cleaning operation, SampleClean applies this operation to a sample, and estimates the result of data analytics. \label{fig:arch}}
\end{SCfigure}

\subsection{Methogology Matters}
From interviews with data scientists in technology companies, we found that many already sample data when prototyping expensive data transformations.
The contribution of SampleClean is to recognize that certain methodologies for sampling and cleaning can lead to inaccurate or even misleading results.
SampleClean provides a framework to apply budgeted data cleaning in a reliable way with provable guarantees on the result.

\vspace{0.5em}
\noindent\textbf{Not Only About Latency: } As applied in SampleClean, random sampling is not only a technique to reduce the latency of data cleaning but also a technique to simultaneously estimate the impact of data error.
When errors are unknown and systematic, the randomization of sampling allows us to bound the estimates.
One of the goals of this project is to develop reliable estimation techniques that will work in a wide variety of error and query regimes.

\vspace{0.5em}
\noindent\textbf{Unbiasedness: } For some types of analytics, we can guarantee that the estimate is unbiased. This means that, in expectation, over all samples of size $k$ the estimate is equal to the true value. Unbiased estimates are useful since a confidence interval measuring the precision of the estimate is equivalent to measuring accuracy w.r.t to the fully cleaned result.

\vspace{0.5em}
\noindent\textbf{Confidence Intervals: } SampleClean bounds every estimate in a confidence interval. This is a significant result since, without data cleaning, the error is potentially unbounded. Bounded estimates (even if loose) give the analyst guidance on how inaccurate an estimate is.

\subsection{Wisteria: Implementing SampleClean \cite{haas2015wisteria}}
We implemented the ideas in SampleClean as a part of a distributed data cleaning library named Wisteria.
Wisteria is built on the Berkeley Data Analytics Stack with numerous optimized data cleaning primitives.
This insight of this project was that the data cleaning process is inherently iterative, with evolving cleaning workflows that 
start with basic exploratory data analysis on small samples of dirty data, then refine analysis with 
more sophisticated/expensive cleaning operators (e.g., crowdsourcing), and finally apply the insights to a full dataset.
Wisteria designed to support the iterative development and optimization of data cleaning workflows, especially ones that utilize the crowd.

In Wisteria, sampling is a first-class logical operator for data cleaning plans that tolerate approximation, and use it to speed up iteration on early-stage plans.
Analysts can prototype expensive data cleaning operations and estimate the effects using samples.
The samples also serve to tune data cleaning plans and a recommendation engine proposes changes to in-flight cleaning plans that allow users to trade off accuracy and latency, and the system provides efficient mechanisms for implementing recommended changes without re-executing the plan on already cleaned tuples.

\begin{figure}[t]
\centering
\vspace{-0.5cm}
\includegraphics[width = .4\textwidth]{figs/lifecycle.png}
\vspace{-0.4cm}
\caption{Example iterations on the design of the portion of a cleaning plan that extracts restaurant addresses from their unstructured webpages.  
1) An exploratory plan that uses a sample to evaluate a simple address extraction method.
2) A plan that applies the method to the entire dataset. The quality is unsatisfactory. 
3) An alternate plan that uses manual crowd extraction. The quality is now high, but the crowd-based extractor is slow. 
4) A hybrid plan that sends only difficult webpages to the crowd, maximizing accuracy without sacrificing latency.}
\label{fig:ex-plan}
\vspace{-0.3cm}
\end{figure}




\fontsize{8.4pt}{8.7pt} \selectfont
\bibliographystyle{abbrv}
\bibliography{ref} 

\end{document}

