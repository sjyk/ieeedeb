\section{Introduction}
Data are susceptible to various forms of corruption such as missing,
incorrect, or inconsistent representations \cite{Gartner}.
Real-world data is commonly integrated from multiple sources, and the integration process may lead to a variety of data errors~\cite{DBLP:journals/pvldb/DongS13}. 
Data analysts report that data cleaning remains one of the most time
consuming steps in the analysis process \cite{nytimes}.
Identifying and fixing data error often requires manually inspecting data, which can quickly become costly and time-consuming. 
While crowdsourcing is an increasingly viable option for some types of errors ~\cite{DBLP:conf/sigmod/JefferyFH08,DBLP:journals/pvldb/FanLMTY10,DBLP:journals/pvldb/YakoutENOI11, gokhale2014corleone, park2014crowdfill, sampleclean,chu2015katara}, it comes at the significant cost of additional latency and the overhead of managing human workers. 

Ignoring the effects of dirty data is potentially dangerous.
Since \emph{a priori}, the magnitude of data error is unknown, any amount of error can be biasing.
Analysts have to choose between facing the cost of data cleaning
or coping with consequences of unknown inaccuracy.
In this article, we describe a middle ground that we call SampleClean \cite{wang1999sample, krishnan2015svc}; where an analyst can clean a sample of data, and use this sample to improve an erroneous query result.
The intriguing part of SampleClean is that while the query result is computed using a sample,                                   
results can often be more accurate than those computed over the full data due to the data cleaning.
This approximation error is boundable unlike the unknown data error, and the tightness of the bound is parametrized by a flexible cleaning cost (i.e., the sampling ratio).

The case for SampleClean is analogous to the case for Approximate Query Processing \cite{DBLP:conf/icde/OlkenR92, olken1993random, garofalakis2001approximate, AgarwalMPMMS13} (AQP).
For decision problems, exploratory analysis problems, and visualization, it often suffices to return an approximate query result bounded in confidence intervals; thus saving significant processing costs.
In many common aggregates, the samples-to-accuracy trade off tends to vary as $O(\frac{1}{\sqrt{n}})$, and therefore every additional $\epsilon$ factor of accuracy costs quadratically more. 
In applications where approximation can be tolerated, sampling avoids the expensive ``last mile" of processing and timely answers facilitate improved user experiences and faster analysis.

In traditional AQP, approximation necessarily sacrifices accuracy for reduced latency. 
However, the goal of SampleClean differs from AQP, as SampleClean trades off data transformation cost for gradual improvements in query accuracy through data cleaning.
While SampleClean introduces approximation error, the data cleaning mitigates errors in query results.
There is a break-even point where a sufficient amount of data is cleaned to facilitate an accurate approximation of queries on the cleaned data, and in this sense, sampling actually improves the accuracy of the query result.

SampleClean \cite{wang1999sample} and all of its extensions \cite{krishnan2015svc}, work in the \emph{budgeted data cleaning} setting. 
An analyst is allowed to apply an expensive data transformation $C(\cdot)$ to only $k\ll N$ rows in a relation.
One solution could be to draw $k$ records uniformly at random and apply data cleaning, e.g., a direct extension of AQP to the cleaned sample.
However, data cleaning presents a number of methodological problems that make this hard.
First, $C(\cdot)$ may change the sampling statistics, for example, duplicated records are more likely to be sampled.
Next, query processing on partially clean data, i.e., a mix of dirty and clean data, can lead to unreliable results due to the well known Simpsons Paradox.
Finally, high-dimensional analytics such as Machine Learning may be very sensitive to sample size, perhaps even more so than to dirty data, and techniques for avoiding sample size dependence are required.

One of the key insights is that there are two contrasting query processing techniques to address every budgeted data cleaning problem.
A direct extension of AQP can estimate the true query result based on the cleaned sample (possibly with some reweighting to account for changes in sampling statistics). 
A sample of cleaned data can also be used to correct the error in a query result over the dirty data.
There is an interesting theoretical tradeoff between these approaches, where the first approach is \emph{robust} as its accuracy is independent of the magnitude of data error, and the second approach is \emph{sample-efficient} as its accuracy is less dependent on sample size.

By establishing the link between AQP and data cleaning, this idea has opened up a number of new research opportunities. 
We applied the same principles to other domains with expensive data transformations such as Materialized View Maintenance and Machine Learning.
In this article, we highlight three projects:

\vspace{0.5em}
\noindent \textbf{\sampleclean \cite{wang1999sample}: } \sampleclean estimates aggregate (\sumfunc, \countfunc, \avgfunc) queries using samples of clean data. \sampleclean reweights the data to compensate for changes in sampling statistics such that the estimates are unbiased and bounded in confidence intervals.

\vspace{0.5em}
\noindent \textbf{View Cleaning \cite{krishnan2015svc}: } View Cleaning generalizes the notion of data cleaning to include expensive maintenance of out-of-date data. Staleness in materialized views (MVs) manifests itself as data error, i.e., a stale view has missing, superfluous, and incorrect rows.
Like data cleaning, eager MV maintenance is expensive, and View Cleaning models querying an MV as a budgeted data cleaning problem.
Aggregate queries are approximated from a stale MV using a small sample of up-to-date data, resulting in bounded estimates.

\vspace{0.5em}
\noindent \textbf{ActiveClean: } ActiveClean extends \sampleclean to a class of analytics problems called Convex Data Analytics (subsuming the aggregates studied in \sampleclean and including Machine Learning such as Support Vector Machines and Linear Regression). ActiveClean exploits the convex structure of the problem (i.e. gradients and feature space) to prioritize cleaning data that is likely to affect the model. ActiveClean directly integrates cleaning into the model training loop and as a result gives a bounded approximation for any cleaning budget.

\vspace{0.5em}

This article is organized as follows. Section 2 introduces the project and its main ideas. Section 4/5/6 describes \sampleclean, View Cleaning, and ActiveClean respectively. Section 7 reviews the related work in this field. In Section 8, we highlight some of the open problems and future directions of the SampleClean project. Finally, we conclude in Section 9.